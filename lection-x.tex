\documentclass[aspectratio=169]{beamer}
\usepackage[utf8]{inputenc}
\usepackage[english,russian]{babel}
\usepackage{cancel}
\usepackage{amssymb}
\usepackage{stmaryrd}
\usepackage{cmll}
\usepackage{graphicx}
\usepackage{amsthm}
\usepackage{tikz}
\usepackage{multicol}
\usetikzlibrary{patterns}
\usepackage{chronosys}
\usepackage{proof}
\usepackage{multirow}
\usepackage{comment}
\setbeamertemplate{navigation symbols}{}
%\usetheme{Warsaw}

\newtheorem{thm}{Теорема}[section]
\newtheorem{dfn}{Определение}[section]
\newtheorem{lmm}{Лемма}[section]
\newtheorem{exm}{Пример}[section]
\newtheorem{snote}{Пояснение}[section]

\newcommand{\divisible}%
{\mathrel{\lower.2ex%
\vbox{\baselineskip=0.7ex\lineskiplimit=0pt%
\kern6pt \hbox{.}\hbox{.}\hbox{.}}%
}}

\begin{document}

\newcommand\doubleplus{+\kern-1.3ex+\kern0.8ex}
\newcommand\mdoubleplus{\ensuremath{\mathbin{+\mkern-10mu+}}}

\begin{frame}{}
\begin{center}\Large О лабах\end{center}
\end{frame}

\begin{frame}{C. Нормализация метаязыка}
Доказательство на метаязыке:
$$\begin{array}{l}\Gamma_1 \vdash \alpha_1\\\Gamma_2 \vdash \alpha_2\\\dots\\\Gamma_n \vdash \alpha_n\end{array}$$
Требуется перестроить его в доказательство в гильбертовском стиле:
$$\begin{array}{l}\Gamma_n\vdash\alpha_n\\\alpha'_1\\\dots\\\alpha'_k\\\alpha_n\end{array}$$
\end{frame}

\begin{frame}{Общая идея перестроения}
Возможные типы переходов:
\begin{enumerate}
\item Аксиома $\Gamma \vdash \alpha$ --- формула $\alpha$ корректен в итоговом доказательстве без изменений.
\item Modus Ponens $$\begin{array}{l}\Gamma \vdash \alpha_1\\\Gamma\vdash\alpha_1\rightarrow\alpha_2\\\Gamma\vdash\alpha_2\end{array}$$
Переход остаётся корректен также без изменений, главное --- следить за совпадением списка $\Gamma$ (с учётом возможных перестановок).
\item Дедукция $$\begin{array}{l}
    \Gamma,\gamma_1,\gamma_2\vdash\alpha_1\rightarrow\alpha_2\rightarrow\beta\\
    \Gamma,\gamma_2,\alpha_2\vdash\gamma_1\rightarrow\alpha_1\rightarrow\beta
\end{array}$$
Данный переход представляет главный интерес.
\end{enumerate}
\end{frame}

\begin{frame}{Посмотрим внимательней}
Вспомним теорему о дедукции с конструктивным доказательством: 
\begin{center}$\Gamma,\alpha\vdash\beta$ эквивалентно
$\Gamma\vdash\alpha\rightarrow\beta$\end{center}
Делается это путём перестроения всего вывода (помните, гипотезы изменяются,
потому теорема затронет формулы вывода и при переносе $\alpha$ влево).

\begin{exm}[$\Gamma,\gamma_1,\gamma_2\vdash\alpha_1\rightarrow\alpha_2\rightarrow\beta\Rightarrow\Gamma,\gamma_2,\alpha_2\vdash\gamma_1\rightarrow\alpha_1\rightarrow\beta$]
Пусть дан вывод $\mathcal{B}_1$, показывающий $\Gamma,\gamma_1,\gamma_2\vdash\alpha_1\rightarrow\alpha_2\rightarrow\beta$.
\begin{enumerate}
\item По $\mathcal{B}_1$ построим $\mathcal{B}_2: \Gamma,\gamma_1,\gamma_2,\alpha_1\vdash\alpha_2\rightarrow\beta$
\item По $\mathcal{B}_2$ построим $\mathcal{B}_3: \Gamma,\gamma_1,\gamma_2,\alpha_1,\alpha_2\vdash\beta$
\item По $\mathcal{B}_3$ построим $\mathcal{B}_4: \Gamma,\gamma_1,\gamma_2,\alpha_2\vdash\alpha_1\rightarrow\beta$
\item По $\mathcal{B}_4$ построим $\mathcal{B}_5: \Gamma,\gamma_2,\alpha_2\vdash\gamma_1\rightarrow\alpha_1\rightarrow\beta$
\end{enumerate}
\end{exm}

Если не думать об оптимизации, то каждый шаг предпологает построение копии всего вывода с изменениями.
\end{frame}

\begin{frame}{Идеи об ускорении вывода}
\begin{itemize}
\item Хранить доказательство в виде дерева.
\item Применение дедукционного перехода может быть сделано как особый узел в дереве доказательства.
\item При перестроении преобразовывать дерево лениво:
\begin{itemize}
\item идти вверх по дереву, от итоговой формулы;
\item игнорировать неиспользуемые узлы;
\item суммировать преобразования при возможности (нейтрализовать встречные дедукции --- аккуратно! преобразования не коммутируют;
накапливать последовательные и т.п.).
\end{itemize}
\item Не требуется идеальное решение --- достаточно прохождения тестов жюри.
\end{itemize}
\end{frame}

\begin{frame}{E. Свобода для подстановки}
Задача хорошо решается через последовательное применение унификации несколько раз.

Выражение в алгебраических термах: $$\theta ::= f_k(\theta_1,\dots,\theta_n)\ |\ x_i$$
В нашем случае $f_k$ --- это и логические и предметные символы. Переменные --- предметные переменные.

Например, если $(\forall x.\varphi) \rightarrow \psi$ есть 11 аксиома, то
$$\pi = \mathcal{U}\big[\varphi,\psi\big]$$ существует и либо тривиально, либо содержит единственную замену --- для
переменной $x$. 
\end{frame}

\begin{frame}{Эквивалентные преобразования системы уравнений}
\begin{thm}
Пусть $\mathcal{E} = \{ \sigma_1=\tau_1, \sigma_2=\tau_2, \dots, \sigma_k=\tau_k \}$ --- система 
уравнений в алгебраических термах. Тогда каждое из следующих преобразований оставляет множество
решений системы неизменным:
\begin{enumerate}
\item убрать уравнение вида $\sigma_t=\sigma_t$ из системы;
\item заменить уравнение вида $f(\theta_1, \dots, \theta_j) = f(\theta'_1, \dots, \theta'_j)$ на
семейство уравнений $\theta_1=\theta'_1; \dots; \theta_j=\theta'_j$;
\item сделать подстановку при наличии уравнения вида $x = f(\theta_1, \dots, \theta_j)$: заменить
все остальные вхождения $x$ в систему на $f(\theta_1, \dots, \theta_j)$;
\item поменять выражения в уравнении местами: $\sigma_t=x_j$ на $x_j=\sigma_t$.
\end{enumerate}
\end{thm}
\end{frame}

\begin{frame}{Решение системы уравнений}
\begin{thm}
Преобразования системы уравнений (при исключении преобразований, ведущих к повторению системы) 
за конечное время приведут:
\begin{enumerate}
\item либо к системе вида $x_i = \sigma_i$, причём каждая из переменных $x_i$
входит в систему ровно один раз;
\item либо к несовместной системе: такой, в которую входит уравнение вида $$x_i = \dots x_i \dots$$
Здесь переменная $x_i$ нетривиально входит справа от знака равенства.
\item также, несовместная система получится, если в ней появится уравнение вида $$f(\theta_1,\dots,\theta_k) = g(\eta_1,\dots,\eta_l)$$
%Понятно, что нет никакой подстановки, превращающей один функциональный символ в другой.
\end{enumerate}
\end{thm}

После применения преобразований по системе можно будет построить требуемую подстановку.
\end{frame}

\begin{frame}{Оптимизации}
В принципе, не обязательно делать унификацию как указано выше. Но нужно трезво понимать, что вы делаете --- иначе 
отладка будет сложна :)
\end{frame}

\begin{frame}{O. Ординальный калькулятор}
В задаче будем применять определение сложения, умножения и умножения через $\underline{sup}(X) = \bigcup X$.
За переосмысление решения для определения операций с лекции --- доп. баллы.
\begin{dfn}Канторовой нормальной формой назовём выражение вида
$$\omega^{\beta_1}\cdot a_1 + \omega^{\beta_2}\cdot a_2 + \dots + \omega^{\beta_n} \cdot a_n$$
причём $\beta_1 > \beta_2 > \dots > \beta_n \ge 0$ и $a_i \in \mathbb{N}, a_i > 0$.
\end{dfn}

\begin{thm}Для любого ординала $\alpha < \varepsilon_0$ существует единственная КНФ, равная ему.\end{thm}
\end{frame}

\begin{frame}{Некоторые свойства операций}
\begin{thm}Для операций над ординалами выполнены следующие свойства:
\begin{enumerate}
\item $\alpha\cdot(\zeta_1+\zeta_2) = \alpha\cdot\zeta_1 + \alpha\cdot\zeta_2$ (левая дистрибутивность)
\item $\alpha\cdot 0 = 0$
\item $\alpha^{\beta_1}\cdot\alpha^{\beta_2} = \alpha^{\beta_1+\beta_2}$
\item $(\alpha^\beta)^\gamma = \alpha^(\beta\cdot\gamma)$
\item $\alpha^0 = 1$
\end{enumerate}
\end{thm}
\begin{proof}Трансфинитная индукция по структуре\end{proof}
\end{frame}

\begin{frame}{Сложение КНФ}
Заметим, что $$\omega^{\beta_1}\cdot a_1 + \omega^{\beta_2}\cdot a_2 = \left\{\begin{array}{ll}
\omega^{\beta_1}\cdot a_1 + \omega^{\beta_2}\cdot a_2, & \beta_1 > \beta_2\\
\omega^{\beta_1}\cdot (a_1 + a_2), & \beta_1 = \beta_2\\
\omega^{\beta_2}\cdot a_2, & \beta_1 < \beta_2\end{array}\right.$$
Имея это свойство, можно вычислить сложение двух КНФ так, чтобы результатом также была некоторая КНФ.
\end{frame}

\begin{frame}{Умножение КНФ}
Заметим, что при $\xi > 0$ 
$$(\omega^{\beta_1}\cdot a_1 + \dots + \omega^{\beta_n}\cdot a_n) \cdot \omega^{\xi} = \omega^{\beta_1 + \xi}$$
И также, если $x\in\mathbb{N}, x > 0$, то
$$(\omega^{\beta_1}\cdot a_1 + \omega^{\beta_2}\cdot a_2 + \dots + \omega^{\beta_n}\cdot a_n) \cdot x = \omega^{\beta_1}\cdot (a_1 \cdot x) + \omega^{\beta_2}\cdot a_2\dots + \omega^{\beta_n}\cdot a_n$$
\end{frame}

\begin{frame}{Возведение в степень}
Если $\gamma$ --- предельный, и $\beta_1 > 0$, то
$$(\omega^{\beta_1}\cdot a_1+\omega^{\beta_2}\cdot a_2+\dots+\omega^{\beta_n}\cdot a_n)^\gamma = (\omega^{\beta_1})^\gamma$$
\end{frame}

\begin{frame}{Идея решения задачи}
С помощью определённых операций сложения, умножения и возведения в степень для канторовых нормальных форм 
возможно перевести любое ординальное выражение, использующее только $\omega$, числа и операции сложения,
умножения и возведения в степень, в КНФ, после чего эти КНФ сравнить.
\end{frame}
\end{document}
